\chapter{Das Projekt}\index{chp\_Das_Projekt}

\section{Systemrecorder}
\subsection{Aufgabenstellung}
Erarbeitung einer Möglichkeit die Daten nativ auf dem Linux Gerät, als auch im Webbrowser graphisch aufzubereiten, wobei die Konformität an Stilrichtlinien gegeben sein muss.
Des Weiteren sollte erreicht 
\subsection{Bisheriger Stand}
\subsection{Recherche möglicher Lösungen}
\subsubsection{WebAssembly - Was ist das?}
Webassembly ist ein vom W3C festgelegter Standard, der einen Bytecode definiert, der in Webbrowsern aber auch außerhalb ausgeführt werden kann. \\
Die Version 1.0 wurde im März 2017 veröffentlicht und wird seitdem von allen gängigen Browser Engines unterstützt. \\
Unterstützt wird Webassembly in Chrome seit Chrome 57, in Firefox seit Firefox 52, in Edge seit Edge 16 und in Safari seit Safari 11. \\
\paragraph*{Vorteile}

\paragraph*{Nachteile}
\begin{itemize}
    \item   Aktuell maximal 4 Gibibytes Speicher adressierbar, es gibt ein Proposal, welches aktuell in Entwicklung ist, das zu beheben und dem
            Webassembly Module die Möglichkeit zu geben 32-Bit oder 64-Bit Speicher Indizes zu nutzen.
\end{itemize}
\paragraph*{Analyse der Technologie}

\subsubsection{Tooling : Emscripten}
\subsubsection{Tooling : Bibliothek : SDL}
\subsection{Durchführung}
\subsubsection{Aufbau des Buildsystems}
\subsection{Gewünschte Funktionen}
\subsection{Interaktion mit Legacy Code}
\subsection{Wesentlicher Code}
%%umbennennes dummer name

\section{Dauertestanlage}
\subsection{Aufgabenstellung}
\subsection{Bisheriger Stand}
\subsection{Durchführung}
\subsection{Hardware}
\subsection{Software}

