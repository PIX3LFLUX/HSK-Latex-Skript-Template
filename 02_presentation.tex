\documentclass[t, aspectratio=169]{beamer}

% To have print without pause
% \documentclass[handout, t, aspectratio=169]{beamer}

\usepackage[utf8]{inputenc}
\usepackage{hyperref}

% Change the presentation layout
\usetheme{Madrid}

% Change the caption size of figure
\usepackage[font=small,skip=4pt]{caption}
\usepackage[symbol]{footmisc}

% Font
\usepackage{plex-sans}

\usepackage[style=numeric,backend=biber,
            doi=false,isbn=false,url=false,eprint=false]{biblatex}
\usepackage{latexsym,xcolor,multicol,booktabs,calligra}
\usepackage{amssymb,amsfonts,amsmath,amsthm,mathrsfs,mathptmx}
\usepackage{graphicx,pstricks,listings,stackengine}
% \usefonttheme[onlymath]{serif}

\usepackage[caption=false]{subfig}
\usepackage{tabularx}
\usepackage{booktabs}

\usepackage{color, colortbl}

% Define color for highlighted table
\definecolor{TableHighLight}{RGB}{190, 250, 196}

\setbeamertemplate{navigation symbols}{}
% \setbeamertemplate{footline}[frame number]
\setbeamertemplate{bibliography item}{\insertbiblabel}
% \setbeamertemplate{footnote}{\insertfootnotetext}

\let\oldfootnotesize\footnotesize
\renewcommand*{\footnotesize}{\oldfootnotesize\tiny}

\DeclareCiteCommand{\footpartcite}[\mkbibfootnote]
{\usebibmacro{prenote}}%
{\usebibmacro{citeindex}%
    \mkbibbrackets{\usebibmacro{cite}}%
    \setunit{\addnbspace}
    \printnames{labelname}%
    \setunit{\labelnamepunct}
    \printfield[citetitle]{title}%
    \newunit
    \printfield[]{year}}
{\addsemicolon\space}
{\usebibmacro{postnote}}

\setbeamertemplate{footnote}{\insertfootnotetext}
\setbeamertemplate{caption}[numbered]

% \usepackage[style=numeric, sorting=none]{biblatex}
 % Insert the structure.tex file which contains the majority of the structure behind the template

%------------------------------------------------------------
%This block of code defines the color theme
\definecolor{ThemeColor}{RGB}{ 40,135, 50} % Elektro- und Informationstechnik
%\definecolor{ThemeColor}{RGB}{ 25,130,130} % Architektur- und Bauwesen
%\definecolor{ThemeColor}{RGB}{ 40,105,175} % Maschinenbau und Mechatronik
%\definecolor{ThemeColor}{RGB}{ 30, 70,150} % Wirtschaftswissenschaften
%\definecolor{ThemeColor}{RGB}{100, 55,140} % Informatik und Wirtschaftsinformatik
%\definecolor{ThemeColor}{RGB}{140, 45,130} %Informationsmanagement und Medien
\usecolortheme[named=ThemeColor]{structure}

%------------------------------------------------------------
%This block of code defines the information to appear in the
%Title page
\title[Short Title] %optional
{Presentation Title}

\subtitle{Sub-title}

\author[Last Name, First name] % (optional)
{Last name, First Name \\
first.lastname@h-ka.de}

\date[April 27th, 2023] % (optional)
{April 27th, 2023}


%End of title page configuration block
%------------------------------------------------------------


%------------------------------------------------------------
%The next block of commands puts the table of contents at the 
%beginning of each section and highlights the current section:

\AtBeginSubsection[]
{
  \begin{frame}
    \frametitle{Table of Contents}
    \tableofcontents[currentsection, currentsubsection]
  \end{frame}
}

% If you need only the Section displayed at Table of Contents
% uncomment this and comment the above
% \AtBeginSubSection[]
% {
%   \begin{frame}
%     \frametitle{Table of Contents}
%     \tableofcontents[currentsubsection]
%   \end{frame}
% }
%------------------------------------------------------------



\makeatletter
\def\@makefnmark{}
\makeatother


\addbibresource{bibliography.bib}





\begin{document}

%The next statement creates the title page.
\frame{\titlepage}


%---------------------------------------------------------
%This block of code is for the table of contents after
%the title page
\begin{frame}
\frametitle{Table of Contents}
\tableofcontents
\end{frame}
%---------------------------------------------------------


\section{Einleitung}
\begin{frame}
\frametitle{Einleitung}

\begin{itemize}
    \item Introduction 1
    \item Introduction 2
\end{itemize}

\end{frame}

\section{Section 1}
\subsection{Sub-section 1.1}

\begin{frame}
\frametitle{Insert image figure}
\textbf{Purpose:} 
\begin{itemize}
    \item Referencing Figure \ref{fig:placeholder} in-text automatically.

\begin{figure}[h]
    \centering
    \includegraphics[scale=0.5]{Pictures/placeholder.jpg}
    \caption{Figure caption\textsuperscript{*}.}
    \label{fig:placeholder}
\end{figure}

\footnotetext{\textsuperscript{*} Footnote text}

\end{itemize}
\end{frame}


\begin{frame}
\frametitle{Bullet points - pause}
\begin{itemize} [<+->]
    \item Use this setting at the itemsize to give a pause between bullet points.
    \item To cite a reference \cite{book_key}. 
    \item Or add `pause` command to have a pause by yourself.
\end{itemize}

\pause

Use minipage environment to have multiple images in the same figure (Fig. \ref{fig:multiple}).

\begin{figure}
\begin{minipage}{.7\textwidth}
\centering
\subfloat[A]{\includegraphics[width=.3\linewidth]{Pictures/placeholder.jpg}}~%
\subfloat[B]{\includegraphics[width=.3\linewidth]{Pictures/placeholder.jpg}}~
\subfloat[C]{\includegraphics[width=.3\linewidth]{Pictures/placeholder.jpg}}

\caption{Multiple images.}
\label{fig:multiple} 
\end{minipage}%

\end{figure}

\vspace{-1.8em}

\footpartcite{book_key}
\end{frame}

\subsection{Sub-section 1.2}
\begin{frame}
\frametitle{Table}

Create a table (Table \ref{table:highlight})

\begin{table}[!h]
\small
\begin{center}
\begin{tabular}{||c | c c | c c c||} 
 \hline
 \# & A & B & C \textsuperscript{\dag} & D \textsuperscript{\S} & E\textsuperscript{\dag\dag} \\ [0.5ex] 
 \hline\hline
 1 & A & B & C & D & E \\ 
 \hline
 2 & A & B & C & D & E \\ 
 \hline
\rowcolor{TableHighLight}
 5 & A & B & C  & D & E \\
 \hline
 	
\rowcolor{TableHighLight}
 6 & A & B & C & D & E \\ [0.2ex] 
 \hline
\end{tabular}
\caption{\label{table:highlight} Table with highlights }
\end{center}
\end{table}

\footnote{\textsuperscript{\dag} The complexity of the method: number of processes, parameters, etc.}
\footnote{\textsuperscript{\S} How well does the method work under different circumstances}
\footnote{\textsuperscript{\dag\dag} How much data we need for the method}

\end{frame}

\section{Conclusion}
\begin{frame}{Conclusion}
\end{frame}

\section{Q\&A}
\begin{frame}
\centering \Huge
\emph{Q\&A}

\end{frame}

\begin{frame}

\centering \Huge
\emph{Thank You}

\end{frame}

%---------------------------------------------------------

% \begin{frame}[t,allowframebreaks]{References} %% Aligned top
% % \input{ms.bbl}
% % \printbibliography[heading=none]
% \end{frame}


\end{document}