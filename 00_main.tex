%%%%%%%%%%%%%%%%%%%%%%%%%%%%%%%%%%%%%%%%%
%
% Important note:
% Chapter heading images should have a 2:1 width:height ratio,
% e.g. 920px width and 460px height.
%
% The original template (the Legrand Orange Book Template) can be found here --> http://www.latextemplates.com/template/the-legrand-orange-book
%
% Original author of the Legrand Orange Book Template:
% Mathias Legrand (legrand.mathias@gmail.com) with modifications by: Vel (vel@latextemplates.com)
%
% Original License:
% CC BY-NC-SA 3.0 (http://creativecommons.org/licenses/by-nc-sa/3.0/)
%%%%%%%%%%%%%%%%%%%%%%%%%%%%%%%%%%%%%%%%%

%----------------------------------------------------------------------------------------
%	PACKAGES AND OTHER DOCUMENT CONFIGURATIONS
%----------------------------------------------------------------------------------------

\documentclass[11pt,fleqn,openany]{book} % Default font size and left-justified equations

\input{structure.tex} % Insert the structure.tex file which contains the majority of the structure behind the template

%----------------------------------------------------------------------------------------
%	VARIOUS REQUIRED PACKAGES AND CONFIGURATIONS
%----------------------------------------------------------------------------------------

\usepackage{graphicx} % Required for including pictures
\graphicspath{{Pictures/}} % Specifies the directory where pictures are stored

\usepackage{lipsum} % Inserts dummy text

\usepackage{tikz} % Required for drawing custom shapes

\usepackage[german]{babel} % german language/hyphenation

\usepackage{enumitem} % Customize lists
\setlist{nolistsep} % Reduce spacing between bullet points and numbered lists

\usepackage{booktabs} % Required for nicer horizontal rules in tables

\usepackage{xcolor} % Required for specifying colors by name
%\definecolor{ThemeColor}{RGB}{ThemeColor} % is done in main, as it depends on faculty

%TODO wollen wir das so haben? Finde ich persönlich besser - Marcel
\usepackage[section]{placeins} % Graphics and tables are forced to be set before the beginning of a new section. Althoug this does not work with subsections. But you can use \FloatBarrier before a subsection to prevent this.

%----------------------------------------------------------------------------------------
% User created custom commands
%----------------------------------------------------------------------------------------

%\usepackage{float}
% 'OneFIGure' erzeugt eine Figure mit einem Bild darin. 
% Nutzung: \ofig{caption}{image_filename (ohne Dateiendung)}{image_width}
\newcommand{\ofig}[3]{
	\begin{figure}[h]
		\centering\includegraphics[width=#3]{#2}
		\caption{#1}
		\label{fig:#2}
	\end{figure}
}




% 'PlaceholderFIGure' erzeugt Figure mit dem Platzhalterbild darin. 
% Nutzung: \pfig{string caption}
\newcommand{\pfig}[1]{
	\ofig{#1}{placeholder}{\linewidth}
}





% 'Create TikZ Titlepage' creates the Title Page and changes the colour scheme corresponding to the faculty
% Nutzung: \createTikZTitlePage{Text: Faculty}{Text: Title}{Text: Authors}
\newcommand{\createTikZTitlePage}[3]{
	\begingroup
	\thispagestyle{empty} % Suppress headers and footers on the title page
	\begin{tikzpicture}[remember picture,overlay]

	%%% node coordinates A4 %%%
		% specify absolute placement on page, figures will be drawn relative to those
		\coordinate (UoAS) at ($ (current page.north west) + (2,-2) $);
		\coordinate (Faculty) at ($ (UoAS) + (0,-2) $);
		\coordinate (Title) at ($ (current page.north west) + (2,-11.5) $);
		\coordinate (Authors) at ($ (current page.south west) + (2,1.5) $);
		\coordinate (BackgroundPic) at ($ (current page.south) + (0,3) $);
		\coordinate (ColorBlock) at (0,-16);
		\coordinate (H) at ($ (current page.north east) + (8,-0.8) $);
		\coordinate (K) at ($ (H) + (0,-18) $);
		\coordinate (A) at ($ (K) + (0,-8.6) $);
		
	%%% Background Picture %%%
		\node[above, inner sep=0pt] (Hinterlegbild) at (BackgroundPic)
		% upper picture boundary needs to be at 3/5 of Page, which is 17.82cm. It starts 3cm from bottom so: 14.82cm
		{\includegraphics[height=14.82cm]{../Pictures/MUM_IP65_IP30.png}}; 
		
	%%% Colored, 3cm high, Box under Picture %%%
		\path[fill=ThemeColor] (current page.south west) rectangle ($ (current page.south east) + (0,3) $);
		
		%%% Text Blocks %%%
		% HKA, University of Applied Science
		\node[text width = 10cm, ThemeColor, below right] at (UoAS) {\avantfont{\textbf{Hochschule Karlsruhe}}\\ \avantfont{University of}\\ \avantfont{Applied Sciences}};
		% Faculty
		\node[text width = 10cm, ThemeColor, below right] at (Faculty) {\avantfont{Fakultät für}\\ \avantfont{\textbf{#1}}};
		% Title and Subtitle 
		\node[text width = 10cm, ThemeColor, above right] at (Title) {#2};
		% Authors
		\node[text width = 10cm, white, right] at (Authors) {{\large #3}};
		
		%%% +| KA %%%
		% overlay HKA picture to draw the nodes CAUTION!!! TikzEdt has for whatever reason no permission to edit files unter windows in ...\Program Files\ directory. 
		% If the picture is not loading, try moving .text and .png file in another directory, e.g. \Downloads\
		% ++ refers to the Node last mentioned before. + does the same thing, without setting the current node as new "last mentiones node"
		\path (H) [fill=ThemeColor, scale = 0.45, rounded corners = 1] +(0,-2.95)  -- +(3,-2.95) -- +(3,0) -- +(4.2,0)  -- +(4.2,-2.95) -- +(7.2,-2.95) -- +(7.2,-3.95) -- +(4.2,-3.95) -- +(4.2,-6.9) -- +(3,-6.9) -- +(3,-3.95) -- +(0,-3.95) -- cycle;
		\path (H) [fill=ThemeColor, scale = 0.45, rounded corners = 1] +(7.8,0)  -- +(9,0) -- +(9,-6.9) -- +(7.8,-6.9) -- cycle;
		\path (K) [fill=ThemeColor, scale = 0.45, rounded corners = 1] +(3, 0) -- +(4.2,0) -- +(4.2,-2.8) -- +(9,0) -- +(9,-1.3) -- +(5.2,-3.45) -- +(9,-5.6) -- +(9,-6.9) -- +(4.2,-4.1) -- +(4.2,-6.9) -- +(3,-6.9) -- cycle;
		\path (A) [fill=ThemeColor, scale = 0.45, rounded corners = 1] +(6, 0) -- +(6.65,0) -- +(10.05,-6.9) -- +(8.75,-6.9) -- +(6,-1.3) -- +(4.8,-3.7) -- +(7.52,-3.7) -- +(8.85,-4.8) -- +(4.3,-4.8) -- +(3.25,-6.9) -- +(1.95,-6.9) -- +(5.35,0) -- cycle;
		
		
	%%% might be usefull later: %%%
		% custom font
		% \node[text width = 10cm, green, below right] (v1) at (0,3) {\fontfamily{uncl}\selectfont{\textbf{Hochschule Karlsruhe}\\University of\\Applied Science}};
		%\path (4,5) [fill=black, scale = 1, rounded corners = 3] +(0, 0) -- +(4,0) -- +(4,-4) -- +(0,-4) -- cycle;
		
		%\def \scl_factor_ttp {1};
		% ($ (\scale_factor *2,0) + (Title) $)
		
	\end{tikzpicture}
	\vfill
	\endgroup
}

 % Insert the customcommands.tex file which contains the custom made commands, functions, etc.

\hypersetup{pdftitle={Praxissemesterbericht},pdfauthor={Florian Ittensohn}} % Fill out to include PDF metadata for the author and title of the book
\date{\today}

		
%----------------------------------------------------------------------------------------
%	Color Style (depending on Faculty)
%----------------------------------------------------------------------------------------
% please uncomment the desired color:

\definecolor{ThemeColor}{RGB}{ 40,135, 50} % Elektro- und Informationstechnik
%\definecolor{ThemeColor}{RGB}{ 25,130,130} % Architektur- und Bauwesen
%\definecolor{ThemeColor}{RGB}{ 40,105,175} % Maschinenbau und Mechatronik
%\definecolor{ThemeColor}{RGB}{ 30, 70,150} % Wirtschaftswissenschaften
%\definecolor{ThemeColor}{RGB}{100, 55,140} % Informatik und Wirtschaftsinformatik
%\definecolor{ThemeColor}{RGB}{140, 45,130} %Informationsmanagement und Medien


\begin{document}

%----------------------------------------------------------------------------------------
%	TITLE PAGE
%----------------------------------------------------------------------------------------
\createTikZTitlePage{Elektro- und \\ Informationstechnik}   {{\huge\textbf{Praxissemester Siemens AG}}{\\Praxissemesterbericht - Firmwareentwicklung WLAN und Mobilfunk}}   {Florian Ittensohn, Matrikelnummer 66817}

%----------------------------------------------------------------------------------------
%	COPYRIGHT PAGE
%----------------------------------------------------------------------------------------

% \newpage
% ~\vfill
% \thispagestyle{empty}

% \noindent Copyright \copyright\ 2021 NAME \\ % Copyright notice

% \noindent \textsc{Published by NAME}\\ % Publisher

% \noindent \textsc{book-website.com}\\ % URL

% \noindent
% Dieses Skript wird als Lehrmaterial für die Vorlesung XYZ an der Hochschule Karlsruhe verwendet.
% Das Skript wird unter der folgenden Lizenz veröffentlicht: CC BY-NC-SA 3.0 (http://creativecommons.org/licenses/by-nc-sa/3.0/). \\
% Das originale Template (the Legrand Orange Book Template) kann hier heruntergeladen werden: http://www.latextemplates.com/template/the-legrand-orange-book 

% \noindent \textit{First printing, Feb 2021} % Printing/edition date

%----------------------------------------------------------------------------------------
%	TABLE OF CONTENTS
%----------------------------------------------------------------------------------------

%\usechapterimagefalse % If you don't want to include a chapter image, use this to toggle images off - it can be enabled later with \usechapterimagetrue

% \chapterimage{Pictures/csm_HKA_FK-EIT_2017-9363_a9e6e1d039.jpg} % Table of contents heading image

\pagestyle{empty} % Disable headers and footers for the following pages


\tableofcontents % Print the table of contents itself

\cleardoublepage
\addcontentsline{toc}{section}{\listfigurename}\listoffigures

\cleardoublepage
\addcontentsline{toc}{section}{\listtablename}\listoftables
% \cleardoublepage
% \addcontentsline{toc}{section}{\lstlistoflistings}\lstlistoflistings

% \cleardoublepage % Forces the first chapter to start on an odd page so it's on the right side of the book

\pagestyle{fancy} % Enable headers and footers again


%----------------------------------------------------------------------------------------
%  Teil 1
\part{Praxissemesterbericht - Firmwareentwicklung Siemens AG}
%----------------------------------------------------------------------------------------

%----------------------------------------------------------------------------------------
%	CHAPTER 1
%----------------------------------------------------------------------------------------
\chapterimage{Pictures/csm_HKA_Image_2017_EIT_9006_92a30c9926.jpg} % Chapter heading image
%----------------------------------------------------------------------------------------
\chapter{Einleitung}\index{chp\_Einleitung}
%----------------------------------------------------------------------------------------
Ein großes Problem der allgegenwärtigen Digitalisierung in allen Bereichen ist der Transport der Informationen von Ort A nach Ort B.
Um dieses Problem zu lösen gibt es zahlreiche drahtgebundene und drahtlose Übertragungswege, wobei ich mich hier speziell für die drahtlose Übertragung interessiere.
Aus diesem Grund habe ich mich für ein Praxissemester in der Entwicklung für Kommunikationsgeräte entschieden. \\

In der Zeit vom 01.10.2022 bis einschließlich 31.03.2023 habe ich mein Praktikum bei der Siemens AG, Standort Karlsruhe Knielingen absolivert.
Hier war ich in einer Abteilung tätig, welche die Firmware für Ethernet, Mobilfunk- und WLAN Endgeräte mit Fokus auf das industrielle Umfeld produziert. 

Für das Praktikum habe ich mich entschieden, da es mich besonders gereizt bei einer Anwendung der Hochfrequenztechnik mitzuwirken und diesen Bereich näher kennen zu lernen.
Außerdem wollte ich einen Einblick in die Softwareentwicklung und die verwendeten Prozesse in einem großen Unternehmen bekommen. 
Besonders reizvoll ist hier auch das Zusammenspiel von physikalischen Beschränkungen, das Zusammenspiel der verschiedenen Komponenten im Gerät und die Erstellung von Lösungen für reale Probleme der Kunden und deren Anwendungsgebiete.
Außerdem hat mich hier die geringe Latenzzeit und hohe Datenrate und vor allem die Ausfallsicherheit der verwendeten Geräte interessiert. \\

Bei meinem Praktikum ergab sich für mich die Möglichkeit den Arbeitsalltag in einer Entwicklungsabteilung in einem Großkonzern kennenzulernen. Ich bekam spannende Einblicke in den Alltag in der Firmwareentwicklung und der damit verbunden Testautomatisierung und Validierung.


Auf den nächsten Seiten werde ich ausführlich auf die Tätigkeiten innerhalb meines Praktikums eingehen. 




%----------------------------------------------------------------------------------------
%	Part 2: Format Examples (to be deleted in final/release version)
% \part{TEIL 2}
%----------------------------------------------------------------------------------------

\chapterimage{Pictures/csm_HKA_FK-MMT_Imagefoto_9847_0f123dde52.jpg} % Chapter heading image

\chapter{Formatierungen und Templates}

\section{Table}\index{Table}

\begin{table}[h]
\centering
\begin{tabular}{l l l}
\toprule
\textbf{Treatments} & \textbf{Response 1} & \textbf{Response 2}\\
\midrule
Treatment 1 & 0.0003262 & 0.562 \\
Treatment 2 & 0.0015681 & 0.910 \\
Treatment 3 & 0.0009271 & 0.296 \\
\bottomrule
\end{tabular}
\caption{Table caption}
\label{tab:example} % Unique label used for referencing the table in-text
%\addcontentsline{toc}{table}{Table \ref{tab:example}} % Uncomment to add the table to the table of contents
\end{table}

Referencing Table \ref{tab:example} in-text automatically.


%------------------------------------------------

\section{Citation}\index{Citation}

This statement requires citation \cite{article_key}; this one is more specific \cite[162]{book_key}.

%------------------------------------------------

\section{Lists}\index{Lists}

Lists are useful to present information in a concise and/or ordered way\footnote{Footnote example...}.

\subsection{Numbered List}\index{Lists!Numbered List}

\begin{enumerate}
\item The first item
\item The second item
\item The third item
\end{enumerate}

\subsection{Bullet Points}\index{Lists!Bullet Points}

\begin{itemize}
\item The first item
\item The second item
\item The third item
\end{itemize}

\subsection{Descriptions and Definitions}\index{Lists!Descriptions and Definitions}

\begin{description}
\item[Name] Description
\item[Word] Definition
\item[Comment] Elaboration
\end{description}

This is an example of theorems.

\subsection{Several equations}\index{Theorems!Several Equations}
This is a theorem consisting of several equations.

\begin{theorem}[Name of the theorem]
In $E=\mathbb{R}^n$ all norms are equivalent. It has the properties:
\begin{align}
& \big| ||\mathbf{x}|| - ||\mathbf{y}|| \big|\leq || \mathbf{x}- \mathbf{y}||\\
&  ||\sum_{i=1}^n\mathbf{x}_i||\leq \sum_{i=1}^n||\mathbf{x}_i||\quad\text{where $n$ is a finite integer}
\end{align}
\end{theorem}

\subsection{Single Line}\index{Theorems!Single Line}
This is a theorem consisting of just one line.

\begin{theorem}
A set $\mathcal{D}(G)$ in dense in $L^2(G)$, $|\cdot|_0$. 
\end{theorem}

%------------------------------------------------

\section{Definitions}\index{Definitions}

This is an example of a definition. A definition could be mathematical or it could define a concept.

\begin{definition}[Definition name]
Given a vector space $E$, a norm on $E$ is an application, denoted $||\cdot||$, $E$ in $\mathbb{R}^+=[0,+\infty[$ such that:
\begin{align}
& ||\mathbf{x}||=0\ \Rightarrow\ \mathbf{x}=\mathbf{0}\\
& ||\lambda \mathbf{x}||=|\lambda|\cdot ||\mathbf{x}||\\
& ||\mathbf{x}+\mathbf{y}||\leq ||\mathbf{x}||+||\mathbf{y}||
\end{align}
\end{definition}

%------------------------------------------------

\section{Notations}\index{Notations}

\begin{notation}
Given an open subset $G$ of $\mathbb{R}^n$, the set of functions $\varphi$ are:
\begin{enumerate}
\item Bounded support $G$;
\item Infinitely differentiable;
\end{enumerate}
a vector space is denoted by $\mathcal{D}(G)$. 
\end{notation}

%------------------------------------------------

\section{Remarks}\index{Remarks}

This is an example of a remark.

\begin{remark}
The concepts presented here are now in conventional employment in mathematics. Vector spaces are taken over the field $\mathbb{K}=\mathbb{R}$, however, established properties are easily extended to $\mathbb{K}=\mathbb{C}$.
\end{remark}

%------------------------------------------------

\section{Corollaries}\index{Corollaries}

This is an example of a corollary.

\begin{corollary}[Corollary name]
The concepts presented here are now in conventional employment in mathematics. Vector spaces are taken over the field $\mathbb{K}=\mathbb{R}$, however, established properties are easily extended to $\mathbb{K}=\mathbb{C}$.
\end{corollary}

%------------------------------------------------

\section{Propositions}\index{Propositions}

This is an example of propositions.

\subsection{Several equations}\index{Propositions!Several Equations}

\begin{proposition}[Proposition name]
It has the properties:
\begin{align}
& \big| ||\mathbf{x}|| - ||\mathbf{y}|| \big|\leq || \mathbf{x}- \mathbf{y}||\\
&  ||\sum_{i=1}^n\mathbf{x}_i||\leq \sum_{i=1}^n||\mathbf{x}_i||\quad\text{where $n$ is a finite integer}
\end{align}
\end{proposition}

\subsection{Single Line}\index{Propositions!Single Line}

\begin{proposition} 
Let $f,g\in L^2(G)$; if $\forall \varphi\in\mathcal{D}(G)$, $(f,\varphi)_0=(g,\varphi)_0$ then $f = g$. 
\end{proposition}

%------------------------------------------------

\section{Examples}\index{Examples}

This is an example of examples.

\subsection{Equation and Text}\index{Examples!Equation and Text}

\begin{example}
Let $G=\{x\in\mathbb{R}^2:|x|<3\}$ and denoted by: $x^0=(1,1)$; consider the function:
\begin{equation}
f(x)=\left\{\begin{aligned} & \mathrm{e}^{|x|} & & \text{si $|x-x^0|\leq 1/2$}\\
& 0 & & \text{si $|x-x^0|> 1/2$}\end{aligned}\right.
\end{equation}
The function $f$ has bounded support, we can take $A=\{x\in\mathbb{R}^2:|x-x^0|\leq 1/2+\epsilon\}$ for all $\epsilon\in\intoo{0}{5/2-\sqrt{2}}$.
\end{example}

\subsection{Paragraph of Text}\index{Examples!Paragraph of Text}

\begin{example}[Example name]
\lipsum[2]
\end{example}

%------------------------------------------------

\section{Exercises}\index{Exercises}

This is an example of an exercise.

\begin{exercise}
This is a good place to ask a question to test learning progress or further cement ideas into students' minds.
\end{exercise}

%------------------------------------------------

\section{Problems}\index{Problems}

\begin{problem}
What is the average airspeed velocity of an unladen swallow?
\end{problem}

%------------------------------------------------

\section{Vocabulary}\index{Vocabulary}

Define a word to improve a students' vocabulary.

\begin{vocabulary}[Word]
Definition of word.
\end{vocabulary}


%------------------------------------------------

\section{Figure}\index{Figure}

\begin{figure}[h]
\centering\includegraphics[scale=0.5]{Pictures/placeholder.jpg}
\caption{Figure caption}
\label{fig:placeholder} % Unique label used for referencing the figure in-text
%\addcontentsline{toc}{figure}{Figure \ref{fig:placeholder}} % Uncomment to add the figure to the table of contents
\end{figure}

Referencing Figure \ref{fig:placeholder} in-text automatically.

%----------------------------------------------------------------------------------------
%	BIBLIOGRAPHY
%----------------------------------------------------------------------------------------

\chapter*{Bibliography}
\addcontentsline{toc}{chapter}{\textcolor{ThemeColor}{Bibliography}} % Add a Bibliography heading to the table of contents

%------------------------------------------------

\section*{Artikel}
\addcontentsline{toc}{section}{Articles}
\printbibliography[heading=bibempty,type=article]

%------------------------------------------------

\section*{Bücher}
\addcontentsline{toc}{section}{Books}
\printbibliography[heading=bibempty,type=book]

\section*{Webseiten}
\addcontentsline{toc}{section}{Online}
\printbibliography[heading=bibempty,type=online]

%----------------------------------------------------------------------------------------
%	INDEX
%----------------------------------------------------------------------------------------

\cleardoublepage % Make sure the index starts on an odd (right side) page
\phantomsection
\setlength{\columnsep}{0.75cm} % Space between the 2 columns of the index
\addcontentsline{toc}{chapter}{\textcolor{ThemeColor}{Index}} % Add an Index heading to the table of contents
\printindex % Output the index

%----------------------------------------------------------------------------------------

\end{document}
