%----------------------------------------------------------------------------------------
%	CHAPTER 1
%----------------------------------------------------------------------------------------
\chapterimage{Pictures/csm_HKA_Image_2017_EIT_9006_92a30c9926.jpg} % Chapter heading image
%----------------------------------------------------------------------------------------
\chapter{Einleitung}\index{chp\_Einleitung}
%----------------------------------------------------------------------------------------
Ein großes Problem der allgegenwärtigen Digitalisierung in allen Bereichen ist der Transport der Informationen von Ort A nach Ort B.
Um dieses Problem zu lösen gibt es zahlreiche drahtgebundene und drahtlose Übertragungswege, wobei ich mich hier speziell für die drahtlose Übertragung interessiere.
Aus diesem Grund habe ich mich für ein Praxissemester in der Entwicklung für Kommunikationsgeräte entschieden. \\

In der Zeit vom 01.10.2022 bis einschließlich 31.03.2023 habe ich mein Praktikum bei der Siemens AG, Standort Karlsruhe Knielingen absolivert.
Hier war ich in einer Abteilung tätig, welche die Firmware für Ethernet, Mobilfunk- und WLAN Endgeräte mit Fokus auf das industrielle Umfeld produziert. 

Für das Praktikum habe ich mich entschieden, da es mich besonders gereizt bei einer Anwendung der Hochfrequenztechnik mitzuwirken und diesen Bereich näher kennen zu lernen.
Außerdem wollte ich einen Einblick in die Softwareentwicklung und die verwendeten Prozesse in einem großen Unternehmen bekommen. 
Besonders reizvoll ist hier auch das Zusammenspiel von physikalischen Beschränkungen, das Zusammenspiel der verschiedenen Komponenten im Gerät und die Erstellung von Lösungen für reale Probleme der Kunden und deren Anwendungsgebiete.
Außerdem hat mich hier die geringe Latenzzeit und hohe Datenrate und vor allem die Ausfallsicherheit der verwendeten Geräte interessiert. \\

Bei meinem Praktikum ergab sich für mich die Möglichkeit den Arbeitsalltag in einer Entwicklungsabteilung in einem Großkonzern kennenzulernen. Ich bekam spannende Einblicke in den Alltag in der Firmwareentwicklung und der damit verbunden Testautomatisierung und Validierung.


Auf den nächsten Seiten werde ich ausführlich auf die Tätigkeiten innerhalb meines Praktikums eingehen.